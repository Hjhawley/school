\documentclass[12pt]{article}
\usepackage[utf8]{inputenc}
\usepackage{amsmath}
\usepackage{amssymb}
\usepackage{geometry}
\usepackage{framed} % For the box

% Adjust margins
\geometry{margin=1in}

\title{\textbf{CSE421: Design and Analysis of Algorithms}}
\author{Homework 1 \\ Shaqan Oveis Gharan}
\date{March 27th, 2024 \\ Due: April 3rd, 2024 at 11:59 PM}


\begin{document}

\noindent CS-3510-(write here your section number) Algorithms, Spring 2025\hfill Homework 1\\
FirstName LastName \hfill Collaborator(s):

\hrulefill

\subsection*{Problem 1 (20 pts)}

Prove or disprove:

\begin{enumerate}
    \item[a)] (10 pts) For any instance of the stable matching problem, the following holds: In \textbf{every} stable matching, there is an \textbf{agent} (a company or an applicant) who gets their first (most favorite) choice. 
    
    \textit{Hint:} If correct, you need to give a proof (be sure to show the claim for every instance and for every stable matching!). If incorrect, give a counter-example, i.e., an instance and a stable matching (be sure to justify that the matching is stable!). In this case, only one counter-example is enough.
    
    \item[b)] (10 pts) For every instance of the stable matching problem, the following holds: There \textbf{exists} a stable matching in which a company gets their first (most favorite) choice.
\end{enumerate}

\subsection*{Problem 2 (10 pts)}

Prove that in every instance of the stable matching problem with $n$ companies and $n$ applicants, companies make at most $n(n-1)+1$ proposals in the Gale-Shapley algorithm.

\textit{Hint:} Prove that in the Gale-Shapley algorithm (when companies propose), at most one company gets its last choice.

\subsection*{Problem 3 (10 pts)}

Use induction to solve this problem: Given $2^k$ real numbers $x_1, \ldots, x_{2^k}$ such that $\sum_{i=1}^{2^k} x_i = 1$, show that:
\[
\sum_{i} x_i^2 \geq \frac{\left(\sum_{i} x_i\right)^2}{2^k} = \frac{1}{2^k}.
\]

\textit{Hint:} You can use the following inequality: For any two real numbers $a, b$, 
\[
(a+b)^2 \leq 2a^2 + 2b^2.
\]

\subsection*{Problem 4 (10 pts)}

Let $I = (P, R)$ be an instance of the stable matching problem. Suppose that the preference lists of all $p \in P$ are identical, so without loss of generality, $p_i$ has the preference list $[r_1, r_2, \ldots, r_n]$. Prove that there is a unique solution to this instance. Describe what the solution looks like, and why it is the only stable solution.

\textit{Note:} Showing that the solution is the one found by the Gale-Shapely algorithm is not sufficient, as there could be other solutions.

\subsection*{Problem 5 (15 bonus pts)}

Implement the Gale-Shapley algorithm to solve the Stable Matching Problem using $O(n^2)$ time complexity. You are free to write in any programming language you like. Make sure that you test your algorithm on small instance sizes, where you are able to check results by hand. Run your algorithm on the following testing instance of size $n=4$ (define the algorithm as a function and give preference matrices as function parameters). Give the resulting matching that is found, along with the list of proposals performed by the algorithm. Submit your source code file.

\begin{itemize}
    \item \textbf{Proposers (P):} $P= \{1, 2, \ldots, n\}$
    \item \textbf{Receivers (R):} $R= \{1', 2', \ldots, n'\}$
    \item \textbf{Preference profiles:}
    \begin{itemize}
        \item Each proposer has a ranked list of receivers.
        \item Each receiver has a ranked list of proposers.
    \end{itemize}
\end{itemize}


\textbf{Function definition:} \texttt{gale\_shapley(P\_pref, R\_pref)} \\

\textbf{Input:}
\begin{itemize}
    \item Two preference matrices:
    \begin{itemize}
        \item \texttt{P\_pref[p][i]:} $i$-th preferred receiver of proposer $p$.
        \item \texttt{R\_pref[r][i]:} $i$-th preferred proposer of receiver $r$.
    \end{itemize}
\end{itemize}

\textbf{Output:}
\begin{itemize}
    \item \texttt{match\_P[p]:} The receiver matched to proposer $p$.
    \item \texttt{match\_R[r]:} The proposer matched to receiver $r$.
\end{itemize}

\textbf{Testing Input:}
\[
\texttt{P\_pref} = 
\begin{bmatrix}
3 & 2 & 4 & 1 \\
1 & 2 & 4 & 3 \\
1 & 2 & 3 & 4 \\
1 & 2 & 3 & 4
\end{bmatrix}
\]
\[
\texttt{R\_pref} = 
\begin{bmatrix}
1 & 3 & 2 & 4 \\
3 & 1 & 4 & 2 \\
4 & 3 & 2 & 1 \\
3 & 4 & 2 & 1
\end{bmatrix}
\]

\textit{Note:} Refer to the lecture slides of week 2, pages 10-11, on the details of efficient implementation. You may switch to 0-based indices if necessary.

\end{document}
