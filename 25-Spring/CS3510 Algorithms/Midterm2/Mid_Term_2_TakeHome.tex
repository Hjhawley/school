\documentclass[12pt]{article}
\usepackage[utf8]{inputenc}
\usepackage{amsmath}
\usepackage{soul}
\usepackage{xcolor}
\usepackage{amssymb}
\usepackage{geometry}
\usepackage{framed} % For the box
\sethlcolor{yellow}
\usepackage{amssymb}
\usepackage{algorithm}
\usepackage{algpseudocode}

% Adjust margins
\geometry{margin=1in}

\begin{document}

\noindent CS-3510-02 Algorithms, Spring 2025\hfill Mid-term 2 take home part\\
Hayden Hawley \hfill Collaborator(s):

\hrulefill

\textbf{Academic Integrity Requirement:} You must write your solution in your own words. Do not copy from external sources, classmates, or generative AI tools.

\subsection*{Problem 1 (10 pts)}

An independent set $I$ in an undirected graph $G = (V, E)$ is a subset $I \subseteq V$ of vertices such that no two vertices in $I$ are joined by an edge of $E$. Consider the following greedy algorithm to try to find a maximum size independent set which is based on the general idea that choosing vertices with small degree to be in $I$ will rule out fewer other vertices:

\begin{algorithm}
\caption{GreedyIndependentSet}
\begin{algorithmic}[1]
\State $I \gets \emptyset$
\While{$G$ is not empty}
    \State Choose a vertex $v$ of smallest degree in $G$ \Comment{Not counting deleted edges}
    \State $I \gets I \cup \{v\}$
    \State Delete $v$ and all its neighbors and their connected edges from $G$\\
    \Comment{None of the neighbors can be included since $v$ is included}
\EndWhile
\State \textbf{return} $I$
\end{algorithmic}
\end{algorithm}

Prove that this algorithm is incorrect by counterexample. Specifically, give an example of a graph on which this algorithm does not produce a largest size independent set. Show both the independent set that the algorithm finds and a larger independent set.

\textbf{Solution:}

The greedy algorithm does NOT always find the biggest possible independent set. Consider this graph:

\begin{verbatim}
    A   D
     \ /
      B
     / \
    C   E
\end{verbatim}

Edges: A-B, B-C, B-D, B-E

\begin{itemize}
  \item All outer nodes (A, C, D, E) have degree 1. B has degree 4.
  \item Greedy would pick A first (lowest degree).
  \item It then removes A and B.
  \item We're left with C, D, and E. The final independent set is {C, D, E}.
\end{itemize}

But the actual largest solution is {A, C, D, E}. 4 nodes, and none are connected.
Therefore, this greedy algorithm picked a smaller set than was possible.

\subsection*{Problem 2 (10 pts)}
Suppose $A$ is an array of $n$ integers that is a strictly decreasing sequence, followed by a strictly increasing sequence such as $[12, 9, 8, 6, 3, 4, 7, 9, 11]$. Give an $O(\log n)$ algorithm to find the minimum element of the array. Justify your algorithm is correct.

\textbf{Solution:}

This is a simple binary search problem. You can find the lowest point by recursively splitting the array in half.

\begin{enumerate}
    \item Start with the whole array, which can really be thought of as two concatenated pre-sorted arrays. We're trying to find the value which is smaller than the values to its left AND right.
    \item Look at the middle (index len($A$)/2) element:
    \begin{itemize}
        \item If it's \textbf{greater} than the element after it, the valley is to the \textbf{right}. Throw away the left half.
        \item If it's \textbf{smaller} than the next element, check if it's also smaller than the previous element. If so, we found the minimum.
        \item Otherwise, the valley is to the \textbf{left}. Throw away the right half.
    \end{itemize}
    \item Repeat until we find the minimum.
\end{enumerate}

Since we halve the array at each step, the runtime is $O(\log n)$ like any binary search problem.

\subsection*{Problem 3 (10 pts)}
Consider a graph that is a path, where the vertices are $v_1, v_2, \ldots, v_n$, with edges between $v_i$ and $v_{i+1}$. Suppose that each node $v_i$ has an associated weight $w_i$. Give an algorithm that takes an $n$-vertex path with weights and returns an independent set of maximum total weight. The runtime of the algorithm should be polynomial in $n$. Justify your algorithm is correct.

\textbf{Solution:}

We have a path of weighted nodes, and we want to choose a set of non-adjacent nodes such that the total weight is maximized. We can use dynamic programming to break the problem down.

At each node $i$, we have two options:
\begin{itemize}
    \item \textbf{Skip} the current node and take the best total from $i - 1$
    \item \textbf{Take} the current node and add its weight $w[i]$ to the best total from $i - 2$
\end{itemize}

\textbf{Pseudocode:}
\begin{verbatim}
dp[0] = w[0]
dp[1] = max(w[0], w[1])
for i from 2 to n-1:
    dp[i] = max(dp[i-1], dp[i-2] + w[i])
\end{verbatim}

This algorithm runs in $O(n)$ time.

\end{document}


