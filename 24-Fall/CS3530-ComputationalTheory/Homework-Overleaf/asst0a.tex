%
% Assignment 0a for CS3530 Computational Theory:
% LaTeX
% Fall 2024
%
% Problems taken from Sipser
%

\documentclass{article}

\usepackage[margin=1in]{geometry}
\usepackage{amsfonts}
\usepackage{amsmath}
\usepackage[english]{babel}
\usepackage[utf8]{inputenc}
\usepackage{ae,aecompl}
\usepackage{emp,ifpdf}
\usepackage{graphicx}

\ifpdf\DeclareGraphicsRule{*}{mps}{*}{}\fi

\empprelude{input boxes; input theory}

% skip for paragraphs, don't indent
\addtolength{\parskip}{0.5\baselineskip}
\parindent=0pt

\begin{document}
\begin{empfile}

\begin{center}
\textbf{\Large CS 3530: Assignment 0a} \\[2mm]
Fall 2024

\emph{Hayden Hawley}
\end{center}

\raggedright

\section*{LaTeX Setup (10 points)}

\subsection*{Problem}

%%%Follow the instructions for setting up your GitHub repository and connecting it to Overleaf. Build a PDF file from this source file.
Find and follow the instructions for setting up Overleaf. Build a PDF file from this source file.

\section*{Exercise 0.1a (5 points)}

\subsection*{Problem}

Examine the following formal descriptions of sets so that you understand which members they contain.  Write a short informal English description of each set.

\begin{itemize}
\item[a.] $\{ 1, 3, 5, 7, ...\}$

\textbf{Solution}

All odd positive integers.

\end{itemize}

\section*{Exercise 0.2a (5 points)}

\subsection*{Problem}

Write formal descriptions of the following sets.

\begin{itemize}
\item[a.] The set containing the numbers 1, 10, and 100.

\textbf{Solution}

\{1,10,100\}

\end{itemize}

\end{empfile}
\immediate\write18{mpost -tex=latex \jobname}
\end{document}
